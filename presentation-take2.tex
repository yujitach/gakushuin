%% hack to load amsfonts instead of amssymb to properly make hbar bold
\RequirePackage{scrlfile}
\makeatletter
\AfterPackage{beamerbasemodes}{\beamer@amssymbfalse}
\makeatother
\documentclass[xcolor={svgnames,rgb}]{beamer}
\let\oldhbar\hbar
\usepackage{amssymb}
\def\hbar{\boldsymbol{\oldhbar}}

%%other packages
\usepackage{braket}

%% fonts

\usepackage[no-math]{fontspec}
%\setmainfont[Mapping=tex-text,Numbers={Lining}]{Hoefler Text}
\setmainfont[Mapping=tex-text]{Optima}
\usepackage[SlantFont]{xeCJK}
\setCJKmainfont{Hiragino Maru Gothic Pro}
\setCJKfamilyfont{tt}{Hiragino Maru Gothic Pro}




%% colors

\definecolor{myblue}{HTML}{100080}
\definecolor{math}{HTML}{100080}
\definecolor{blendedblue}{rgb}{0.2,0.2,0.7}

\def\bff{\ifmmode\else\bfseries\fi}
\def\black#1{\textcolor{Black}{#1}}
\def\red#1{\textcolor{Crimson}{\bff #1}}
\def\green#1{\textcolor{ForestGreen}{\bff #1}}
\def\blue#1{\textcolor{myblue}{\bff #1}}
\def\orange#1{\textcolor{Orange}{\bff #1}}
\def\purple#1{\textcolor{Purple}{\bff #1}}
\def\alert#1{\red{#1}}

\everymath{\displaystyle\color{math}}
\let\olddisplaystyle\displaystyle
\def\displaystyle{\olddisplaystyle\color{math}}
\let\oldbracket\[
\def\[{\oldbracket\color{math}}



%% hyperref setup
%\hypersetup{colorlinks=true,urlcolor=Purple}
\let\oldhref\href
%\def\href#1#2{\oldhref{#1}{\purple{#2}}}


%% citation etc
\def\loosecite#1{\textcolor{Purple}{[#1]}}
\def\arxiv#1{\oldhref{http://arxiv.org/abs/#1}{#1}}
\long\def\openquestion#1{%
\begin{block}{Open question}
#1
\end{block}
}



%% beamer setup
\usetheme{Boadilla}
\usecolortheme{default}
\usecolortheme[named=Crimson]{structure}
\setbeamercolor{frametitle}{fg=ForestGreen}
\setbeamercolor{titlelike}{bg=white}
\setbeamercolor{palette primary}{bg=white!70}
\setbeamercolor{palette secondary}{bg=white!80}
\setbeamercolor{palette tertiary}{bg=white!90}
\setbeamercolor{palette quaternary}{bg=white}

\usefonttheme{serif}
\usefonttheme{professionalfonts}
\setbeamerfont{structure}{series=\bfseries}
\setbeamerfont{subsection in toc}{parent=section in toc}

\setbeamertemplate{items}[circle]
\setbeamertemplate{sections/subsections in toc}[sections numbered]
\setbeamertemplate{navigation symbols}{} 


\setbeamertemplate{footline}
{
\leavevmode%
\hbox{%
	\begin{beamercolorbox}[wd=.333333\paperwidth,ht=2.25ex,dp=1ex,center]{author in head/foot}%
	\usebeamerfont{author in head/foot}  \end{beamercolorbox}%
	\begin{beamercolorbox}[wd=.333333\paperwidth,ht=2.25ex,dp=1ex,center]{title in head/foot}%
		\usebeamerfont{title in head/foot}\insertshorttitle
	\end{beamercolorbox}%
	\begin{beamercolorbox}[wd=.333333\paperwidth,ht=2.25ex,dp=1ex,right]{date in head/foot}%
		\usebeamerfont{date in head/foot}\insertshortdate{}\hspace*{2em}%
		\insertframenumber{} / \inserttotalframenumber %
		\hspace*{2ex}%
	\end{beamercolorbox}}%
	\vskip0pt%
}

\let\frametitleoriginal\frametitle
\def\frametitle#1{\frametitleoriginal{#1\vphantom{Hpgy}}}



%% other misc macros
\def\slash#1{\ooalign{\hfil\big/\hfil\crcr$#1$}}
\def\tr{\mathop{\mathrm{tr}}\nolimits}

\def\cL{\mathcal{L}}
\def\bQ{\mathbb{Q}}
\def\bR{\mathbb{R}}
\def\bC{\mathbb{C}}
\def\bZ{\mathbb{Z}}
\def\cN{\mathcal{N}}
\def\cD{\mathcal{D}}
\def\cA{\mathcal{A}}
\def\cF{\mathcal{F}}
\def\cH{\mathcal{H}}
\def\cO{\mathcal{O}}
\def\cS{\mathcal{S}}
\def\cV{\mathcal{V}}
\def\cP{\mathcal{P}}
\def\cM{\mathcal{M}}

\def\vev#1{\langle#1\rangle}
\def\vol{\mathop{\mathrm{vol}}\nolimits}
\def\diag{\mathrm{diag}}
\def\triv{\mathrm{triv}}
\def\SCFT{\mathrm{SCFT}}
\def\Spin{\mathrm{Spin}}
\def\Pin{\mathrm{Pin}}
\def\U{\mathrm{U}}
\def\SU{\mathrm{SU}}
\def\SL{\mathrm{SL}}
\def\SO{\mathrm{SO}}
\def\O{\mathrm{O}}
\def\USp{\mathrm{USp}}
\def\Sp{\mathrm{Sp}}
\def\u{\mathfrak{u}}
\def\su{\mathfrak{su}}
\def\so{\mathfrak{so}}
\def\usp{\mathfrak{usp}}
\def\sp{\mathfrak{sp}}
\def\hkq{/\!/\!/}
\def\RP{\mathbb{RP}}

\def\Nequals#1{$\mathcal{N}{=}#1$}


\subject{Unoriented}
\date[]{}

\usepackage{tikz}
\usepackage{pbox}

\begin{document}

\parskip1em 
\boldmath
\def\baselinestretch{1.1}



\def\inc#1{\vcenter{\hbox{\includegraphics[scale=.15]{#1}}}}
\def\incc#1{\vcenter{\hbox{\includegraphics[scale=.12]{#1}}}}

\begin{frame}
\bigskip\bigskip\bigskip\bigskip\bigskip

\vfill

%\rmfamily

\begin{exampleblock}{}
\begin{center}\LARGE\bfseries
\color{math}
Maxwell 理論の量子異常について
\end{center}
\end{exampleblock}

\bigskip\bigskip\bigskip
\begin{center}
\large  \green{立川裕二}  

\bigskip
\large \blue{第七回統計物理懇談会}

\bigskip
\large 2019年3月6日
\end{center}
\bigskip\bigskip\bigskip
\vfill


\end{frame}



\def\div{\mathop{\mathrm{div}}}
\def\rot{\mathop{\mathrm{rot}}}
\begin{frame}
\begin{align*}
\div \vec B&=\rho_m,  &\rot \vec E + \partial_t \vec B&=\vec J_m, \\
\div \vec E&=\rho_e, & \rot \vec B - \partial_t \vec E&=\vec J_e.
\end{align*}
\begin{center}
とっても重要。
\end{center}

\end{frame}

\begin{frame}
\begin{align*}
\div \vec B&=\rho_m,  &\rot \vec E + \partial_t \vec B&=\vec J_m, \\
\div \vec E&=\rho_e, & \rot \vec B - \partial_t \vec E&=\vec J_e.
\end{align*}

\begin{center}
ガリレイ変換で不変でない! 

$\downarrow$

ローレンツ変換で不変だ!
\end{center}

\end{frame}

\begin{frame}
\begin{align*}
\div \vec B&=\rho_m,  &\rot \vec E + \partial_t \vec B&=\vec J_m, \\
\div \vec E&=\rho_e, & \rot \vec B - \partial_t \vec E&=\vec J_e.
\end{align*}

\begin{center}
電磁双対性: $\vec E\leftrightarrow \vec B$
\end{center}
\end{frame}

\begin{frame}
\begin{align*}
\div \vec B&=\rho_m,  &\rot \vec E \alert{+} \partial_t \vec B&=\vec J_m, \\
\div \vec E&=\rho_e, & \rot \vec B \alert{-} \partial_t \vec E&=\vec J_e.
\end{align*}
\begin{align*}
S: & (\vec E,\vec B) \mapsto (\alert{-}\vec B,\vec E) \\
S^2: & (\vec E,\vec B) \mapsto \alert{-}(\vec E,\vec B)
\end{align*}
\begin{center}
電磁\alert{双対}性は四回やらないと元に戻らない。
\end{center}
\end{frame}

\begin{frame}
\[
\inc{pic1}
\]
\begin{center}
電荷 $q$ と磁荷 $m$ を考える。
\end{center}
\end{frame}


\begin{frame}
\[
\inc{pic2}
\]
\begin{center}
ポインティングベクトル $\vec E\times \vec B$から角運動量 $\propto qm$ が出る。
\end{center}
\end{frame}

\begin{frame}
\[
\inc{pic2}
\]
これまで比例定数を全然決めていなかったので、\[
\text{角運動量} = \frac{\hbar}2 qm
\] としてよい。量子力学では、$qm$ は整数でないといけない。

これが \alert{Dirac 量子化条件}。
\end{frame}

\begin{frame}
\[
\inc{pic2}
\]
電荷と磁荷の単位自体も決めていなかった。\[
\text{角運動量} = \frac{\hbar}2 qm
\]
最小の電荷を $q=1$, 最小の磁荷を $m=1$ と取ることにする。
\end{frame}

\begin{frame}
\[
\inc{pic3}
\]
電荷と磁荷を両方持った粒子を考えても良い。\[
\text{角運動量}=\frac{\hbar}2( \blue{q}\alert{m'}- \blue{m}\alert{q'} ) = \frac{\hbar}2 \det \begin{pmatrix}
q & \alert{q'} \\
m & \alert{m'}
\end{pmatrix}
\]
電場と磁場の変換 \[
\begin{pmatrix}
\vec E \\
\vec B
\end{pmatrix}
\to
\begin{pmatrix}
a & b \\
c & d
\end{pmatrix}
\begin{pmatrix}
\vec E \\
\vec B
\end{pmatrix}
\] は $(q,m)$ にも作用するので、$a,b,c,d$ は整数。

$\det\begin{pmatrix}
q & q' \\
m & m'
\end{pmatrix}
$ を保存するので、$\det\begin{pmatrix}
a & b\\
c & d
\end{pmatrix}=1$。

\end{frame}
\begin{frame}
すなわち\[
\begin{pmatrix}
\vec E \\
\vec B
\end{pmatrix}
\to
\begin{pmatrix}
a & b \\
c & d
\end{pmatrix}
\begin{pmatrix}
\vec E \\
\vec B
\end{pmatrix}
\]
において
\[
\begin{pmatrix}
a & b \\
c & d 
\end{pmatrix} \in SL(2,\mathbb{Z})
\]
特に\[
S: \begin{pmatrix}
\vec E \\
\vec B
\end{pmatrix}
\mapsto
\begin{pmatrix}
-\vec B\\
+\vec E 
\end{pmatrix}
\]は \[
\begin{pmatrix}
a & b\\
c& d
\end{pmatrix}
= \begin{pmatrix}
0 & -1\\
1 & 0
\end{pmatrix}
\]に対応。
\end{frame}

\def\b{\green{b}}
\def\f{\alert{f}}
\begin{frame}
さて、ボゾンとフェルミオンの角運動量は
\[
\begin{array}{c|ccccccc}
\text{\green{boson}} & 0\hbar & & \hbar &&2\hbar & &\cdots\\
\hline
\text{\alert{fermion}} & & \frac{1}2\hbar & & \frac32\hbar && \frac52\hbar & \cdots
\end{array}
\]
だった。\[
\b+\b=\b; \qquad
\b+\f=\f ;\qquad
\f+\f=\b
\]
\end{frame}

\begin{frame}
\green{電荷も磁荷も持たない状態は全てボゾンであるような世界を考える。}

すると\[
\begin{array}{c|cccccccccc}
\text{電荷}& \cdots & -3 & -2 & -1 & 0 & +1 & +2 & +3 & \cdots \\
\hline
\text{$\b$ or $\f$?}& & \b & \b & \b & \b & \b & \b & \b & 
\end{array}
\]か \[
\begin{array}{c|cccccccccc}
\text{電荷}& \cdots & -3 & -2 & -1 & 0 & +1 & +2 & +3 & \cdots \\
\hline
\text{$\b$ or $\f$?}& & \f & \b & \f & \b & \f & \b & \f & 
\end{array}
\]の二通りがありうる。
\end{frame}

\begin{frame}
同様に\[
\begin{array}{c|cccccccccc}
\text{\alert{磁荷}}& \cdots & -3 & -2 & -1 & 0 & +1 & +2 & +3 & \cdots \\
\hline
\text{$\b$ or $\f$?}& & \b & \b & \b & \b & \b & \b & \b & 
\end{array}
\]か \[
\begin{array}{c|cccccccccc}
\text{\alert{磁荷}}& \cdots & -3 & -2 & -1 & 0 & +1 & +2 & +3 & \cdots \\
\hline
\text{$\b$ or $\f$?}& & \f & \b & \f & \b & \f & \b & \f & 
\end{array}
\]
という二通りもある。

\end{frame}

\begin{frame}
\green{電荷も磁荷も持たない状態は全てボゾンであるような世界では、}
\begin{itemize}
\item 電荷 $\equiv 1 \mod 2$  の粒子が $\b$ か $\f$ かの二通り
\item 磁荷 $\equiv 1\mod 2$ の粒子が $\b$ か $\f$ かの二通り
\end{itemize}
で微妙に異なる $2\times 2=4$ 通りの Maxwell 理論がある。

\end{frame}

\begin{frame}
磁荷と電荷を両方持つ粒子の $\b$/$\f$ は? 
\[
\inc{pic4}
\]
角運動量 $\frac{\hbar}2 qm = \frac{\hbar}2$ が追加されることを忘れないようにすると、\\
粒子の $(q,m)$ に応じた $\b$/$\f$ は四通りのそれぞれに対して
\[
\begin{array}{cccccccc}
  (1,0) &+  & (0,1) & \to &  (1,1) \\
 \hline
 \hline
 \b & +&\b & \to &\f \\  
 \b & +&\f & \to &\b \\  
 \f & +&\b & \to &\b \\  
 \hline
 \f & +&\f & \to &\f 
\end{array}
\]
となる。
\end{frame}

\begin{frame}
\[
\begin{array}{cccccccc}
  (1,0) &+  & (0,1) & \to &  (1,1) \\
 \hline
 \hline
 \b & +&\b & \to &\f \\  
 \b & +&\f & \to &\b \\  
 \f & +&\b & \to &\b \\  
 \hline
 \f & +&\f & \to &\f 
\end{array}
\]

\bigskip

$SL(2,\mathbb{Z})$ で $(q,m)\equiv (1,0)$, $(0,1)$, $(1,1)$ は入れ替えられる。

おしまいの一つは $SL(2,\mathbb{Z})$ で不変。

はじめの三つは $SL(2,\mathbb{Z})$ で入れ替わる。$SL(2,\mathbb{Z})$ で不変でない。

($T^2$ 上のスピン構造の振舞いと同じ。)

\end{frame}

\begin{frame}
\[
\inc{pic5}
\]
\begin{center}
三次元空間内の電荷 \quad 四次元時空内の電荷
\end{center}
\end{frame}

\begin{frame}
\[
\inc{pic6}
\]
四次元時空における対称性 $g\in G$ の作用は、

三次元の壁を演算子 $\mathcal{O}$ が越えること、と図示できる。

\end{frame}

\begin{frame}
\[
\inc{pic7}
\]
点演算子 $\mathcal{O}$ でなくて、世界線に作用する「対称性」も考えられる。

一次元の世界線が、二次元の壁を通過すること、と思える。

\end{frame}

\begin{frame}
$d$-次元の対象に作用する対称性を $d$-対称性と呼ぼう。
\[
\begin{array}{cc}
\text{$0$-対称性} & \incc{pic6} \\[5em]
\text{$1$-対称性} & \incc{pic7} 
\end{array}
\]
\end{frame}

\begin{frame}
\[
\inc{pic7}
\] は三次元等時刻面の切り方によってはいろいろに見える:
\[
\inc{pic8}
\]
\[
\inc{pic9}
\]
\end{frame}

\begin{frame}
\[
\inc{pic7}
\]
Maxwell 理論では、次のような二種類の $\mathbb{Z}_2$ $1$-対称性を考えられる:
\begin{itemize}
\item 電気的$1$-対称性:\\
\qquad 電荷 $q$ の世界線を越えると、$(-1)^q$ がその期待値に掛かる
\item 磁気的$1$-対称性:\\
\qquad 磁荷 $m$ の世界線を越えると、$(-1)^m$ がその期待値に掛かる
\end{itemize}
ただし電荷 $q$ の世界線の期待値は Aharanov-Bohm 位相 \[
\exp(2\pi i q\oint \vec A \cdot d\vec x )
\]とする。
\end{frame}

\begin{frame}
\[
\incc{pic7} \, \incc{pic9}
\]
電荷 $q$ の世界線を越えると、$(-1)^q$ がその期待値に掛かる \[
\exp(2\pi i q\oint \vec A \cdot d\vec x )
\mapsto (-1)^q \exp(i q\oint \vec A \cdot d\vec x )
\] すなわち、電気的 $1$-対称性を実現する壁をつくっている黒線は\\
磁束量子の半分 \[
\vec B=\oint  \vec A \cdot d\vec x  = \pm\frac12
\] を持っているということ。

\end{frame}

\begin{frame}
\[
\inc{pic10}
\]
磁荷 $m$ の世界線を越えると、$(-1)^m$ がその期待値に掛かる

これは、磁気的 $1$-対称性を実現する緑の壁には \[
\exp(\pi i \iint \vec B \cdot d\vec\sigma)
\]という因子があるということ。
\end{frame}

\begin{frame}
\begin{gather*}
\text{電気的 $1$-対称性の壁 \qquad 磁気的 $1$-対称性の壁}\\
\qquad\qquad\vec B= \pm\frac12 \qquad\qquad \exp(\pi i \iint \vec B \cdot d\vec\sigma)\\
\inc{pic11}
\end{gather*}
同時に時空に入れると、四次元時空では、二次元面ふたつは点で交わるので微妙なことになる。
\[
\inc{pic12}
\]
\end{frame}

\begin{frame}
一次元おとして絵を書くと、\[
\inc{pic13}
\]
位相因子が \[
e^{\pm\pi i} =\pm i
\] のどちらか定まらない。
\end{frame}

\begin{frame}
場の量子論において対称性を考えた際に、分配関数や真空期待値の位相が定まらなくなる現象を
その対称性の\alert{量子異常}=\alert{アノマリ}という。

もともと1969年代後半にフェルミオンの $U(1)$ 対称性に対して見つかったのがはじまり。

今回のは Maxwell 理論の電気的 $1$-対称性と磁気的 $1$-対称性の間の混合アノマリという変なもの。
\end{frame}

\begin{frame}
さて、電荷/磁荷をもった粒子の $\b$/$\f$ の話と合わせよう。

3+1次元では $720^\circ$ 捻るのは連続的に何も捻らないのと同じだが、
$360^\circ$ 捻るというのは非自明な操作。

ある粒子が fermion であるとは、$360^\circ$ 捻ると $-1$ がつくということ: \[
\inc{pic14}
\]

\end{frame}

\begin{frame}
電荷 $q=1$ を持った粒子が fermion であるとは、$360^\circ$ 捻る操作と
電気的 $1$-対称性の壁で囲むというのが同じであると言うこと。
\[
\inc{pic15}
\]
\end{frame}

\begin{frame}
同様に、磁荷 $m=1$ を持った粒子が fermion であるとは、$360^\circ$ 捻る操作と
磁気的 $1$-対称性の壁で囲むというのが同じであると言うこと。
\[
\inc{pic16}
\]
\end{frame}

\begin{frame}

別に僕は粒子を捻らないからそんなこと気にしないよ、というわけにはいかない。

四次元の時空が非自明だと、時空のパッチ間の貼り合わせのために、$360^\circ$ の捻りが必然的に生じることがある。

これをはかるのが代数トポロジーでいう Stiefel-Whitney 類 $w_2$ というもので、四次元時空 $M_4$ に対して、$360^\circ$ の捻りに対応する $1$-対称性の壁に対応。

複素射影平面 $\mathbb{CP}^2$ だと、$w_2$ はその中の複素射影直線 \[
w_2=\mathbb{CP}^1\subset \mathbb{CP}^2
\] に対応。

\end{frame}

\begin{frame}
さて、Maxwell 理論には四種類あった:

\[
\begin{array}{c|ccccccc}
(q,m) &  (1,0)  & (0,1) &   (1,1) \\
 \hline
 \hline
& \b & \b & \f \\  
& \b & \f & \b \\  
& \f & \b & \b \\  
 \hline
& \f & \f & \f 
\end{array}
\]

\bigskip

$SL(2,\mathbb{Z})$ で $(q,m)\equiv (1,0)$, $(0,1)$, $(1,1)$ は入れ替えられる。

おしまいの一つは $SL(2,\mathbb{Z})$ で不変。

はじめの三つは $SL(2,\mathbb{Z})$ で入れ替わる。$SL(2,\mathbb{Z})$ で不変でない。

\end{frame}

\begin{frame}
\[
\begin{array}{c|ccccccc}
(q,m) &  (1,0)  & (0,1) &   (1,1) \\
 \hline
 \hline
& \b & \b & \f \\  
& \b & \f & \b \\  
& \f & \b & \b \\  
 \hline
& \f & \f & \f 
\end{array}
\]

\bigskip

はじめの $SL(2,\mathbb{Z})$ 不変でない三つは、電気的 $1$-対称性の壁もしくは磁気的 $1$-対称性の壁どちらかしか挿入する必要がないので、何の問題も生じない。

おしまいの $SL(2,\mathbb{Z})$ 不変な一つは、電気的 $1$-対称性の壁および磁気的 $1$-対称性の壁の両方を挿入する必要があるので、$\pm1$ だけの不定性が生じうる。

\end{frame}

\begin{frame}
\[
\begin{array}{c|ccccccc}
(q,m) &  (1,0)  & (0,1) &   (1,1) \\
 \hline \hline
& \f & \f & \f 
\end{array}
\]

おしまいの $SL(2,\mathbb{Z})$ 不変な一つを複素射影平面 $\mathbb{CP}^2$ 上で考えると、\\
フェルミオンの $360^\circ$ 捻りを捉える為、\\
$w_2=\mathbb{CP}^1$ に電気的 $1$-対称性の壁と磁気的 $1$-対称性の壁を\\
それぞれいれないといけない。

\end{frame}

\begin{frame}
$\mathbb{CP}^2$ 内の二つの $\mathbb{CP}^1$ は一ヶ所で交わる
\[
\inc{pic13}
\]
ので、符号が定まらない危険がある。

実際、$\mathbb{CP}^2$  の複素共役写像 $[z:w] \mapsto [\bar z:\bar w]$ のもとで、どうしても符号が反転してしまうことが確認出来る。

\end{frame}

\begin{frame}
まとめると、\[
\begin{array}{c|ccccccc}
(q,m) &  (1,0)  & (0,1) &   (1,1) \\
 \hline \hline
& \f & \f & \f 
\end{array}
\]
に対応する、 $SL(2,\mathbb{Z})$ 不変な Maxwell 理論は、\\
たとえば四次元時空 $\mathbb{CP}^2$ 上で考えると、\\
その座標変換 $[z:w] \mapsto [\bar z:\bar w]$ のもとで分配関数を不変に保てない。

一般共変性の微妙な破れがある。

これは\alert{大域的重力アノマリ}の例。

\loosecite{Wang-Wen-Witten \arxiv{1810.00844} の一部}

\end{frame}

\begin{frame}
アノマリとトポロジカル相には関係がある。
\[
\inc{pic17}
\]
境界が $1+1$ 次元でバルクが $2+1$ 次元の典型例は量子ホール効果。
\end{frame}
\begin{frame}
バルクの有効作用は Chern-Simons 項 \[
\int_{M_3} AdA \propto \int_{M_3} \epsilon^{\mu\nu\rho} A_{\mu} \partial_\rho A_\sigma
\]
ゲージ変換 \[
A\mapsto A+d\chi
\] のもとで \[
\int_{M_3} AdA \mapsto \int_{M_3} AdA + \int_{M_3} d(\chi dA ) = \int_{\partial M_3} \chi dA.
\]
\end{frame}

\begin{frame}
\[
\inc{pic17}
\]
境界でゲージ変換の不変性が \[
\int_{\partial M_3} \chi dA
\]だけ破れている。これと丁度逆の破れをもつ自由度が境界にいないといけない。
\alert{一方向にのみ動くギャップレスフェルミオン}が存在する。
\end{frame}

\begin{frame}
$1+1$ 次元境界の空間方向をさらに円周上において考える。\[
\inc{pic18}
\]
$\varphi:=\int A_x dx$ とすると、エネルギーレベルは $E \propto n + \varphi $

負エネルギー状態は Dirac の海。きちんと正則化すると、Dirac の海は電荷 \[
q = \varphi
\] を持つ。
\end{frame}

\end{document}
